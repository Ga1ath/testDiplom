\documentclass[a4paper, 14pt]{extarticle}
\usepackage{extsizes}
\usepackage[T2A]{fontenc}
\usepackage[utf8]{inputenc}
\usepackage[english, russian]{babel}
\usepackage[fleqn]{amsmath}
\usepackage{amssymb, amsfonts, mathtext, mathtools, cite, enumerate, float}
\usepackage[pdftex]{graphicx}
\usepackage{indentfirst}
\usepackage{listings}
\usepackage{geometry}
\geometry{left=20mm, right=15mm, top=10mm, bottom=20mm}
\parindent = 1.5cm
\usepackage{longtable}
\usepackage{float}

\usepackage{tikz}		%пакеты векторной графики
\usepackage{pgfplots}	%чтобы по набору точек строить диаграммы
\usepackage{xifthen}

\sloppy
\linespread{1.3}
\renewcommand{\rmdefault}{ftm} % Times New Roman
\frenchspacing
\allowdisplaybreaks
\newcommand{\tbs}{\textbackslash}
\newcommand{\bs}{\operatorname{\backslash}}
\newcommand{\listing}{\par\addtocounter{listing_num}{1}%
	\text{Листинг \arabic{listing_num}.} }

\usepackage{color}


\input{/home/me/VII/CourseWork/docs/preproc.tex}
\begin{document}
\section*{Постановка задачи.}

Дано: функция $f(x)$. Нужно найти корень функции $f(x)$ на заданном отрезке $[a,\,b]$ методом касательных. Заранее гарантируется, что на этом отрезке есть только один корень.

\section*{Решение.}

Метод касательных: пусть корень уравнения $f(x)=0$ определен на отрезке $[a,\,b]$, причем $f'(x)$ и $f''(x)$ непрерывны и сохраняют знаки при $x\in[a,\,b]$. Если в достаточно малой окрестности корня $x^*$ уравнения $f(x)=0$ выполняется равенство
$$f(x)=f(x^*)+f'(x^*)\cdot(x-x^*)$$ то решение $x^*$ находится с помощью итерационной процедуры вычисления:
\begin{preproc}
	iter(f,df,x_\text{i}) := x_\text{i} - \frac{f(x_\text{i})}{df(x_\text{i})}\\
\end{preproc}
где $f$, $df$ --- функция и ее производная, $x_\text{i}$ --- $i$-ое приближение $x^*$. В качестве начального приближения выбирается один из концов отрезка $[a,\,b]$, в котором выполняется соотношение $f(x_0)\cdot f''(x_0)>0$.
Критерий окончания счета задается функцией
\begin{preproc}
	stop(f,x_\text{i},x_\text{i+1},e) := f(x_\text{i+1})\cdot(f(x_\text{i+1}+sgn(x_\text{i+1}-x_\text{i})\cdot e)) < 0\\
\end{preproc}
Здесь переменная $e$ задает точность, а функция $sgn(x)$ определена так:
\begin{preproc}
	sgn(x) := \begin{caseblock}
	1 \when x > 0 \\
	0 \when x = 0 \\
	-1 \when x < 0 
\end{caseblock}\\
\end{preproc}
Тогда для нахождения решения уравнения $f(x)=0$ в интервале $[a,\,b]$ будем использовать функцию find, которая возвращает решение и число итераций метода касательных:
\begin{preproc}
	find(a,b,f,df,e) := \begin{block}
		s := (b-a) / \abs(b - a)\\
		i := 1\\
		x_\text{i} := a\\
		x_\text{i+1} := iter(f, df, x_\text{i})\\
		\while{ \neg stop(f, x_\text{i}, x_\text{i+1},e) \land x_\text{i+1} \neq iter(f,df,x_\text{i+1})}\\
		\begin{block}
			i := i+1\\
			x_\text{i} := x_\text{i+1}\\
			x_\text{i+1} := iter(f, df, x_\text{i+1})\\
			\ifexpr{s \neq sgn(b-x_\text{i+1})} x_\text{i} := x_\text{i+1}\\
		\end{block}\\
		\begin{pmatrix}
			x_\text{i+1} & i+1
		\end{pmatrix}\\
	\end{block}\\
\end{preproc}
Решение может быть в точке $a$, $b$ или $t\in(a,\,b)$. В качестве начального приближения выберем точку $a$:
\begin{preproc}
	method(a,b,f,df,e) := \begin{caseblock}
		\begin{pmatrix}a & 0\end{pmatrix} \when f(a) = 0\\
		\begin{pmatrix}b & 0\end{pmatrix} \when f(b) = 0\\
		find(a,b,f,df,e) \otherwise\\
	\end{caseblock}
\end{preproc}
Зададим тестовую функцию, интервал и точность:
