\documentclass[a4paper, 14pt]{extarticle}
\usepackage{extsizes}
\usepackage[T2A]{fontenc}
\usepackage[utf8]{inputenc}
\usepackage[english, russian]{babel}
\usepackage[fleqn]{amsmath}
\usepackage{amssymb, amsfonts, mathtext, mathtools, cite, enumerate, float}
\usepackage[pdftex]{graphicx}
\usepackage{indentfirst}
\usepackage{listings}
\usepackage{geometry}
\geometry{left=20mm, right=15mm, top=10mm, bottom=20mm}
\parindent = 1.5cm
\usepackage{longtable}
\usepackage{float}

\usepackage{tikz}		%пакеты векторной графики
\usepackage{pgfplots}	%чтобы по набору точек строить диаграммы
\usepackage{xifthen}

\sloppy
\linespread{1.3}
\renewcommand{\rmdefault}{ftm} % Times New Roman
\frenchspacing
\allowdisplaybreaks
\newcommand{\tbs}{\textbackslash}
\newcommand{\bs}{\operatorname{\backslash}}
\newcommand{\listing}{\par\addtocounter{listing_num}{1}%
	\text{Листинг \arabic{listing_num}.} }

\usepackage{color}


\input{preproc.tex}
\begin{document}
\section*{Постановка задачи.}

Дано: функция $f(x)$. Нужно найти корень функции $f(x)$ на заданном отрезке $[a,\,b]$ методом касательных. Заранее гарантируется, что на этом отрезке есть только один корень.

\section*{Решение.}

Метод касательных: пусть корень уравнения $f(x)=0$ определен на отрезке $[a,\,b]$, причем $f'(x)$ и $f''(x)$ непрерывны и сохраняют знаки при $x\in[a,\,b]$. Если в достаточно малой окрестности корня $x^*$ уравнения $f(x)=0$ выполняется равенство
$$f(x)=f(x^*)+f'(x^*)\cdot(x-x^*)$$ то решение $x^*$ находится с помощью итерационной процедуры вычисления:
\begin{preproc}
	iter(f,df,x_\text{i}) := x_\text{i} - \frac{f(x_\text{i})}{df(x_\text{i})}\\
\end{preproc}
где $f$, $df$ --- функция и ее производная, $x_\text{i}$ --- $i$-ое приближение $x^*$. В качестве начального приближения выбирается один из концов отрезка $[a,\,b]$, в котором выполняется соотношение $f(x_0)\cdot f''(x_0)>0$.
Критерий окончания счета задается функцией
\begin{preproc}
	stop(f,x_\text{i},x_\text{i+1},e) := f(x_\text{i+1})\cdot(f(x_\text{i+1}+sgn(x_\text{i+1}-x_\text{i})\cdot e)) < 0\\
\end{preproc}
Здесь переменная $e$ задает точность, а функция $sgn(x)$ определена так:
\begin{preproc}
	sgn(x) := \begin{caseblock}
	1 \when x > 0 \\
	0 \when x = 0 \\
	-1 \when x < 0 
\end{caseblock}\\
\end{preproc}
Тогда для нахождения решения уравнения $f(x)=0$ в интервале $[a,\,b]$ будем использовать функцию find, которая возвращает решение и число итераций метода касательных:
\begin{preproc}
	find(a,b,f,df,e) := \begin{block}
		s := sgn(b-a)\\
		i := 1\\
		x_\text{i} := a\\
		x_\text{i+1} := iter(f, df, x_\text{i})\\
		\while{ \neg stop(f, x_\text{i}, x_\text{i+1},e) \land x_\text{i+1} \neq iter(f,df,x_\text{i+1})}\\
		\begin{block}
			i := i+1\\
			x_\text{i} := x_\text{i+1}\\
			x_\text{i+1} := iter(f, df, x_\text{i+1})\\
			\ifexpr{s \neq sgn(b-x_\text{i+1})} x_\text{i} := x_\text{i+1}\\
		\end{block}\\
		\begin{pmatrix}
			x_\text{i+1} & i+1
		\end{pmatrix}\\
	\end{block}\\
\end{preproc}
Решение может быть в точке $a$, $b$ или $t\in(a,\,b)$. В качестве начального приближения выберем точку $a$:
\begin{preproc}
	method(a,b,f,df,e) := \begin{caseblock}
		\begin{pmatrix}a & 0\end{pmatrix} \when f(a) = 0\\
		\begin{pmatrix}b & 0\end{pmatrix} \when f(b) = 0\\
		find(a,b,f,df,e) \otherwise\\
	\end{caseblock}
\end{preproc}
Зададим тестовую функцию, интервал и точность:
\begin{preproc}
	a := 0\\
	b := 1\\
	f(x) := \sin(x)-0.5\\	%функция
	\graphic{f}{\range[0.02]{0}{1}}{(0.000000,-0.500000)
(0.020000,-0.480001)
(0.040000,-0.460011)
(0.060000,-0.440036)
(0.080000,-0.420085)
(0.100000,-0.400167)
(0.120000,-0.380288)
(0.140000,-0.360457)
(0.160000,-0.340682)
(0.180000,-0.320970)
(0.200000,-0.301331)
(0.220000,-0.281770)
(0.240000,-0.262297)
(0.260000,-0.242919)
(0.280000,-0.223644)
(0.300000,-0.204480)
(0.320000,-0.185433)
(0.340000,-0.166513)
(0.360000,-0.147726)
(0.380000,-0.129080)
(0.400000,-0.110582)
(0.420000,-0.092240)
(0.440000,-0.074061)
(0.460000,-0.056052)
(0.480000,-0.038221)
(0.500000,-0.020574)
(0.520000,-0.003120)
(0.540000,0.014136)
(0.560000,0.031186)
(0.580000,0.048024)
(0.600000,0.064642)
(0.620000,0.081035)
(0.640000,0.097195)
(0.660000,0.113117)
(0.680000,0.128793)
(0.700000,0.144218)
(0.720000,0.159385)
(0.740000,0.174288)
(0.760000,0.188921)
(0.780000,0.203279)
(0.800000,0.217356)
(0.820000,0.231146)
(0.840000,0.244643)
(0.860000,0.257843)
(0.880000,0.270739)
(0.900000,0.283327)
(0.920000,0.295602)
(0.940000,0.307558)
(0.960000,0.319192)
(0.980000,0.330497)
}\\
	df(x) := \cos(x)\\		%её производная
	e :=0.01\\
	method(a,b,f,df,e) = \placeholder{\begin{pmatrix}
0.523444 & 3.000000\end{pmatrix}}\\
%	\graphic{f}{\range[0.05]{a}{b}}{}\\
	\end{preproc}
\end{document}

