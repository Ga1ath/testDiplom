\documentclass[a4paper, 14pt]{extarticle}
\usepackage{extsizes}
\usepackage[T2A]{fontenc}
\usepackage[utf8]{inputenc}
\usepackage[english, russian]{babel}
\usepackage[fleqn]{amsmath}
\usepackage{amssymb, amsfonts, mathtext, mathtools, cite, enumerate, float}
\usepackage[pdftex]{graphicx}
\usepackage{indentfirst}
\usepackage{listings}
\usepackage{geometry}
\geometry{left=20mm, right=15mm, top=10mm, bottom=20mm}
\parindent = 1.5cm
\usepackage{longtable}
\usepackage{float}

\usepackage{tikz}		%пакеты векторной графики
\usepackage{pgfplots}	%чтобы по набору точек строить диаграммы
\usepackage{xifthen}

\sloppy
\linespread{1.3}
\renewcommand{\rmdefault}{ftm} % Times New Roman
\frenchspacing
\allowdisplaybreaks
\newcommand{\tbs}{\textbackslash}
\newcommand{\bs}{\operatorname{\backslash}}
\newcommand{\listing}{\par\addtocounter{listing_num}{1}%
\text{Листинг \arabic{listing_num}.} }

\usepackage{color}



\begin{document}

    Функция округления в~меньшую сторону. На~целых числах она иногда врёт.

    \begin{preproc}

        ceil(x) := \begin{block}
                       x := x \cdot \pi - \pi / 2 \\
                       frac := \arctan(\tan(x)) \\
                       (x - frac) / \pi
        \end{block} \\

        ceil(0.4) = \placeholder{} \\
        ceil(0.5) = \placeholder{} \\
        ceil(0.6) = \placeholder{} \\
        ceil(1.0) = \placeholder{} \\
        ceil(1.2) = \placeholder{} \\
        ceil(1.6) = \placeholder{} \\
        ceil(1.8) = \placeholder{} \\
        ceil(2.0) = \placeholder{} \\
        ceil(2.1) = \placeholder{} \\

        \graphic{ceil}{\range[0.1]{0}{5}} {}

    \end{preproc}


\end{document}

