\input{/home/me/VIII/Diplom/docs/preproc.tex}
\begin{document}
    \begin{preproc}

%       Каждый пример ниже работает

        test1() := 2 \\
        test1() = \placeholder{} \\

        x := 2
        y := 3
        test2() := x + y
        test2() = \placeholder{} \\

        test3(x) := x * 2
        test3(7) = \placeholder{} \\

        test4(z) := y * z + x * z
        test4(7) = \placeholder{} \\

        test5(a, b) := test3(a) + test4(b)
        test5(4, 5) = \placeholder{} \\
        test5(test3(x), test4(y)) = \placeholder{} \\

        switch(c) := \begin{caseblock}
                         1 \cdot kg \when x > 0 \\
                         1 \cdot m \when x < 0 \\
                         1 \otherwise
        \end{caseblock} \\ % все ок, так как мы можем узнать, что всегда будет выполняться первая ветка

%       Каждый пример ниже падает с ошибкой, так как статическая типизация не допускает таких конструкций

        test5(c, d) := c(x) + y * d % мы не знаем, какой тип возвращать c(x)

        test6(c, d) := c * d % мы не знаем, какого типа c и d, можно ли их перемножать (вдруг с будет матрицей)

        x := 1 \cdot s
        y := 2 \cdot kg

        z := x + y % разные размерности

        test7(z) := x * y + y * x + z % все ок
        test7(1 \cdot s) = \placeholder{} \\ % z имеет тип сек, а должен сек * кг

    \end{preproc}
\end{document}

